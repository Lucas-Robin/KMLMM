
\section{Introduction}

\subsection{Task}

In this project we compare Kernel Support Vector Machines (KSVMs) and state of the art techniques
like Convolutional Neural Networks (CNNs) for image classification.\\
Since image classification is a multi-class problem,
we implement our own strategies of how to use KSVMs in order to decide which class
will be assigned to a given image, and we compare our strategies to the strategies
which are implemented in the R-library \textit{kernlab}.\\
Both CNNs and KSVMs have a lot of parameters, so a main part of our work
was to search for parameter settings that lead to good classification results.


\subsection{Datasets}

We have three different datasets with varying kinds of images, image sizes and dataset sizes.


\subsubsection{ZIP}

The ZIP dataset was provided in the second part of the course for some homeworks.\\
It contains 16x16-pixel grayscale images of handdrawn digits from 0 to 9.
The dataset is rather small with a training set of 7291 and a test set of 2007 images.

\begin{figure}
 \begin{minipage}{0.32\textwidth}
  \includegraphics[width=\textwidth]{../plots/zip_dataset}
 \end{minipage}
 \begin{minipage}{0.32\textwidth}
  \includegraphics[width=\textwidth]{../plots/mnist_dataset}
 \end{minipage}
 \begin{minipage}{0.32\textwidth}
  \begin{minipage}{0.18\textwidth}
    \includegraphics[width=1.2\textwidth]{../plots/cifar10-class0}
  \end{minipage}
  \begin{minipage}{0.18\textwidth}
    \includegraphics[width=1.2\textwidth]{../plots/cifar10-class1}
  \end{minipage}
  \begin{minipage}{0.18\textwidth}
    \includegraphics[width=1.2\textwidth]{../plots/cifar10-class2}
  \end{minipage}
  \begin{minipage}{0.18\textwidth}
    \includegraphics[width=1.2\textwidth]{../plots/cifar10-class3}
  \end{minipage}
  \begin{minipage}{0.18\textwidth}
    \includegraphics[width=1.2\textwidth]{../plots/cifar10-class4}
  \end{minipage}
  \begin{minipage}{0.18\textwidth}
    \includegraphics[width=1.2\textwidth]{../plots/cifar10-class5}
  \end{minipage}
  \begin{minipage}{0.18\textwidth}
    \includegraphics[width=1.2\textwidth]{../plots/cifar10-class6}
  \end{minipage}
  \begin{minipage}{0.18\textwidth}
    \includegraphics[width=1.2\textwidth]{../plots/cifar10-class7}
  \end{minipage}
  \begin{minipage}{0.18\textwidth}
    \includegraphics[width=1.2\textwidth]{../plots/cifar10-class8}
  \end{minipage}
  \begin{minipage}{0.18\textwidth}
    \includegraphics[width=1.2\textwidth]{../plots/cifar10-class9}
  \end{minipage}
 \end{minipage}
 \caption{Example images of the classes of the ZIP, MNIST and CIFAR10-dataset.}
\end{figure}

\subsubsection{MNIST}

The MNIST dataset is frequently used in the image classification literature
and can be considered to be a standard dataset.
We downloaded the dataset from \cite{mnist}.\\
It contains 28x28-pixel grayscale images of handdrawn digits from 0 to 9.
The dataset is bigger than the ZIP dataset and comes with a training set of 60000 and a test set of 10000 images.


\subsubsection{CIFAR-10}

The CIFAR-10 dataset contains 32x32-pixel RGB-coloured images of the following ten classes:
airplane,
automobile,
bird,
cat,
deer,
dog,
frog,
horse,
ship,
truck.\\

The dataset is divided into a training set of 50000 and a test set of 10000 images.\\
All the data can be found on \cite{cifar10}.
